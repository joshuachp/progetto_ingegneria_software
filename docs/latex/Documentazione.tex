\documentclass[12pt, a4paper]{report}

%package
\usepackage[utf8]{inputenc}
\usepackage{amsmath} %For both in-line and equation mode
\numberwithin{equation}{section} %Numbering of our equations per section
\usepackage{algorithm}
\usepackage{algorithmic} %Algorithm styles, need to be nested for the example shown
\usepackage{fancyhdr} %For our headers
\usepackage{graphicx} %Inserting images
\usepackage{lipsum}  %Blank text fill, delete me when finished
\usepackage{setspace} %Spacing on the front page for crest and titles
\usepackage[]{fncychap} % Styles can be Sonny, Lenny, Glenn, Conny, Rejne, Bjarne and Bjornstrup
\usepackage[hyphens]{url} %Deals with hyphens in urls to make them clickable
\usepackage{xcolor} %Great if you want coloured text
\usepackage{tabularx}
\usepackage{appendix} %Take a wild guess slick

\fancyhf{}
\pagestyle{fancy}

\title{Progetto Ingegneria del Software}
\author{Davide Bleggi \and Joshua Chapman}
\date{Giugno 2020}


\begin{document}
\begin{titlepage}
  \maketitle
\end{titlepage}

% Table of content
\tableofcontents
\newpage

\begin{abstract}
  Progetto per la creazione di un sistema informatico per gestire il servizio
  di spesa on-line di un supermercato.
\end{abstract}

\section{Introduzione}

% Subsection divisione del lavoro, tentativo di scrum, e difficoltà della distanza
% Paired developement
% Cominciato con analisi dei requisiti, sezioni del pdf
% Backlog dei requisiti

% Subsection framework e librerie usati spring per il server e okhttp3. Javafx
% build system gradle

\section{Use cases}

% Use case diagram

% Per ogni use case sequence subsection
\subsection{Sequence diagram}

% Inseriamo grafico
% Spiegazione per punti

\section{Activity diagram}

% Activity diagram e spiegazione

\section{Pattern architetturali}

% MVC Server Repository(Database)

\section{Class diagram}

% Riassunto enf. classi più importanti
% Mod nel database simi modl client non ug

% Sezione server Router

% Sezione model Database
%   model Client

% Classe della Sessione
%   con eventuali robe

% Controllers
% Tutte sequence diagram

% Tasks & Components

\section{Discussione design patterns}

% Singletons
%   Database
%   Sessione
% DAO Observable Sessione
% Factory
%   Utente
% Gli altri non sono factory, ma li abbiamo tenuti cosi per divisione del codice
% e possibile ottimizzazione

\section{Test suite}

% TDD crea test, crea funzione intellij genera metodi
% Unit test tutto server, code coverage
% Javafx non testabile, ma funzioni attorno si
% CI/CD GitHub test automatici

\section{Conclusione}

% Pace e amore schifo java e javafx (╯°□°)╯︵ ┻━┻

\end{document}
